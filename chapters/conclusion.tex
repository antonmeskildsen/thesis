\chapter{Conclusion}
This thesis has presented the concept of an eye information process as a building block for understanding how information is stored in eye tracking data and how different source signals may be selectively degraded by applying obfuscation techniques that exploit specific characteristics of a target signal. This idea makes it possible to redefine eye information process tasks such as gaze estimation and iris recognition using the same terminology. This makes it possible to investigate the interactions between different EIPs using the same theoretical framework and create concepts for information obfuscation that do not depend on specific definitions of individual processes.

Iris obfuscation is used as the focus for the presented experimental work due to the high accuracy of existing recognition methods. Two experiments are proposed to comprehensively test both possible optimisation of method parameters and precise obfuscation performance. A number of new method proposals all outperform existing methods with the proposed method Comb showing by far the best performance. The study is thus meant as the first proper overview of the iris obfuscation problem and how it might be solved. It is also a proposal to use the EIP model for formulating more general problems and work towards strict notions of the obfuscation performance that provide theoretical guarantees instead of only empirical ones.

I propose a number of interesting directions for future development. Using CNNs for iris obfuscation (as proposed by my supervisor) may lead to further analyses of how simple obfuscation methods might work and may be used in the creation of full end-to-end gaze-estimation pipelines using neural networks that implement obfuscation implicitly. Another direction is the use of EIP model and surrounding methodology of obfuscation to EIPs using gaze for information extraction. This use-case is interesting because the information that may be extracted from gaze is very sensitive if identification is possible. Finally, the existing use case and approach (iris obfuscation using simple filters or noise generators) can be improved by automatic parameter optimisation methods and exploration of strict definitions for security using the proposed information measures.

%Eye-tracking is becoming increasingly prevalent due to reductions in production costs and advances in technology and software. Because the human eye can be used to infer different sensitive properties including identity, security and privacy research in eye-tracking is highly relevant. Because the data used in eye-tracking systems contains sensitive information, methods that selectively obfuscate sensitive information need to be developed.


%Security and privacy in eye-tracking are therefore highly relevant areas of research as the potential for abuse or leakage of sensitive 
%Security and privacy in eye-tracking is becoming increasingly important because 


%* A model for understanding eye information
%* A discussion about the importance of defining a common framework for obfuscation.
%* A comprehensive test of iris obfuscation methods
%* A novel method for iris obfuscation that performs very well

%This goal of this project was to look into the question of how sensi- tive information can be removed from eye-tracking processes while retaining utility. Specifically, I investigated how machine learning might help create optimal models for doing so. This led to three sub- questions that I have tried to answer to various degrees. A summary of the answers are presented here in order:
%1. I have presented an overview of the anatomy of eyes as well as the field of eye-tracking and current methods for personal identification using eye-information. This leads me to propose a simple classification scheme for information in eye-trackers. This scheme can be used to easily compare and understand dif- ferent information-related issues in eye tracking.
%2. I have presented a proposal for how sensitive information can be removed without making a negative impact on utility for the specific case of iris-recognition. The proposal is based on general information theory as well as specific methods for iris- recognition. The prototype uses a bilateral filter optimised using random-search to remove iris-information from eye images. The result was a very minor impact on pupil detection accuracy. It is a promising method that deserves further examination.
%3. For security, I presented delentropy as a general measure of in- formation content in signals of highly correlated values such as images. Although it empirically reflects visual degradation of iris-patterns, further testing is needed to assert its actual use- fulness. Additional measures may have to be considered. For utility, the best method may be to test archetypical gaze estima- tion systems for a wide coverage of methods.