\chapter{Future work}
As suggested throughout the thesis, the EIP model and the surrounding work is meant to be an offset for future work in the area of sensitive information removal in eye-tracking. The large number of personally sensitive properties eye-tracking data potentially reveals is a problem that needs to be investigated further. This chapter presents my thoughts on the most critical issues and what solutions might be achievable using the approach presented here.



\section{Iris obfuscation}
Iris-obfuscation is, as mentioned, a good place for foundational research due to the fact that it enables other discovered properties to be linked to an identity with extremely high precision given high-resolution data as is typically recorded with head-mounted eye-trackers. The results presented shows that the proposed model combining blurring and noise generation might be a promising solution. It is however extremely important, that these and other methods are studied in even greater detail to provide better guarantees for their security. 


\subsection{Finding strict boundaries}
I propose to continue pursuing information-related measures as a basis for proving iris obfuscation methods secure. A reasonable starting point is to use the mutual information as well as its ratio to the total entropy of the iris pattern in the image. Alternatively, the capacity measure can be used to define theoretical upper bounds that are not dependent on specific data.

The histograms used in these experiments use only 16 bins per dimension which itself removes a lot of information. The resulting values are therefore not representative of the true entropy of the images. Since this is limited by the number of samples, using a whole dataset as a single signal to estimate the distribution will make it possible to more accurately approximate the true distribution. Other approaches such as kernel density estimation might be used as well.

\subsection{Efficient parameter optimisation}\label{sec:future-optim}
As mentioned in \cref{sec:detail-opt}, a vector evaluated genetic algorithm has been implemented as part of the optimisation system. A vector evaluated genetic algorithm is a simple variation of a traditional genetic algorithm with a selection criteria, crossover, and mutation steps as defined in \parencite[148, 222]{kochenderfer2019algorithms}. Instead of selecting a number of parents based on a single selection criteria, the vector algorithm divides the population into subpopulations which are each optimised (selection) according to a specific output feature. The result is that all measures are considered. 

Due to time constraints and the relative effectiveness of the brute-force grid-search, this implementation was never adequately tested but might be useful, especially for methods with higher numbers of parameters. The Comb and Comb-reverse methods are interesting candidates for this approach due to them having three parameters.

\subsection{Learning iris obfuscation}
Applying neural network methods to learn the obfuscation functions from data has two purposes. (1) Given the success of neural networks in other areas of computer vision, it is reasonable to expect that an obfuscation method trained from data would be better adapted to the task than the methods presented here. (2) The learned models can be analysed to better understand how iris obfuscation can work.

\subsubsection{A possible model}
One of the original goals for this thesis was to create a simple CNN model for iris obfuscation. This section contains my thoughts for a definition. Neural networks are an ideal model for this application since they allow optimisation of arbitrary functions. This means that complex priors and methods for measuring cost can be implemented easily. 
%Neural networks are just differentiable functions which are optimised using stochastic gradient descent. The stochastic modifier signifies, that instead of determining the true gradient, a smaller sample is used instead, resulting in realistic time-frames for optimisation. Although neural networks are often described by layers and neurons, this terminology is all a combination of the history of their creation and useful architectural terms. In general, we may model any function as long as it can be differentiated.

For iris obfuscation, an important characteristic is effectiveness since the method should work in an embedded environment either as software or hardware. Locality and depth are therefore important considerations for the design of a neural network for iris obfuscation.

Because both the input and output are images, the architecture should be a fully convolutional network (FCN) which contains no linear layers. This allows the "field of view" of each pixel to be limited by the radius
\begin{align}
    neigh = (k-1)/2*L,
\end{align}
where $k$ is the kernel size and $L$ is the number of layers. 

The gradient-based entropy measures are an ideal candidate for measuring the degree of obfuscation in an image since it is differentiable. Alternatively, neural-network based iris recognition models may be used \parencite{nguyen2017iris, gangwar2016deepirisnet} but they introduce method-specific biases that may or may not be acceptable. Additionally, the measures presented in the article are shown to correlate highly with the iris recognition accuracy.

%Redefining the measures as differentiable functions require a histogram binning method that is continuous. A method proposed in \parencite{avi2019hue} is extended to be usable for multivariate distributions. 
%
%The traditional histogram counting function is defined as
%\begin{align}
%	h(v) = \sum_{x\in I} \delta_{I_x, v},
%\end{align}
%which is not continuous. Instead, the counting function is substituted with 
%
%\begin{align}
%    \Pi_k(z) = \sigma(\frac{z-\mu_k+L/2}{W}) - \sigma(\frac{z-\mu_k-L/2}{W})
%\end{align}
%
%\begin{align}
%    P_I(k) = \frac{1}{N}\sum_{x\in\Omega}\Pi_k(I(x))
%\end{align}
%
%\begin{align}
%    P_{\nabla I}(\partial x, \partial y) = \frac{1}{N}\sum_{x\in\Omega}\Pi_{\partial x}(I(x))\Pi_{\partial y}(I(x)) = \frac{1}{N}P_{\partial x}P_{\partial y}^T
%\end{align}
%
%\begin{align}
%    P_{\nabla I_1, \nabla I_2}(\partial x_1, \partial y_1, \partial x_2, \partial y_2) = \frac{1}{N}\sum_{x\in\Omega}\Pi_{\partial x_1}(I(x))\Pi_{\partial y_1}(I(x))\Pi_{\partial x_2}(I(x))\Pi_{\partial y_2}(I(x))
%\end{align}
%
%\begin{align}
%    J(\nabla I_1, \nabla I_2) = \frac{1}{N}P_1P_2^TP_3P_4^T
%\end{align}
%
%\begin{figure}
%    \centering
%    \begin{tikzpicture}
%    \begin{axis}
%    \addplot[
%        samples=100
%    ]
%    {1/(1+exp(-x)};
%    \end{axis}
%    \end{tikzpicture}
%    \caption{Caption}
%    \label{fig:my_label}
%\end{figure}
%
%
%\begin{align}
%    J(\theta) = \frac{1}{2}\mathbb{E}_{x,y\sim \hat{p}_{data}} ||\mathbf{y}-f(x;\theta)||^2
%\end{align}
%
%\begin{align}
%    Z_{i,j,k} = \sum_{l,m,n} V_{l,j+m-1, k+n-1}K_{i,l,m,n}
%\end{align}


\todo{kun hvis du kan nå det!!}

\subsection{Combined training}
I propose that training a complete gaze estimation pipeline as neural network with a dedicated iris obfuscation component is an obvious next step for exploring how the iris and gaze signal components in images might be modified in a manner that is beneficial to both iris obfuscation and gaze estimation. In other words, I suggest the possibility of improving neural-network based gaze estimation by jointly training a network for iris obfuscation and gaze estimation. Although the encoded iris is clearly indicative of the eye's position and rotation and thus usable for gaze estimation, a jointly trained method may selectively retain important features for gaze while removing noise and iris pattern information.





