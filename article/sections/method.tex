\section{Obfuscation methods methods}

The gaussian filter is a low-pass filter, which means precisely that it allows low-frequency components of a signal to pass while it blocks high-frequency components. This is ideal for obfuscating iris patterns which are high frequency, but also lead to the deterioration of the large eye features (although they may be recoverable in other ways).

The same problem is present in noise removal which is why gaussian filters are only used sparingly. Instead, a number of non-linear filters which act as neighbourhood aware low-pass filters, can be used... bilateral + non-local-...

Another strategy entirely is to add extremely high-frequency noise to the images. This is an extremely cheap operation and is not even dependent on an image neighbourhood. The limitation here is the possibility of noise removal techniques, especially if multiple, almost identical, images can be obtained, which allows for exploitation of statistical methods to recreate the original signal.



Finally, we might consider a different approach entirely where the nature of the filter is determined by optimisation. Much more about this...

\subsubsection{Gaussian filter}
\begin{equation*}
    GB[I](p) = \sum_{q\in S} G_{\sigma_s}(||p-q||)I_q ,
\end{equation*}


\subsubsection{Bilateral filter}
\begin{equation*}
    BF[I](p) = \frac{1}{W_p}\sum_{q\in S} G_{\sigma_s}(||p-q||)G_{\sigma_r}(|I_p - I_q|)I_q ,
\end{equation*}

\begin{equation*}
    W_p = \sum_{q\in S} G_{\sigma_s}(||p-q||)G_{\sigma_r}(|I_p - I_q|) 
\end{equation*}

\subsubsection{Non-local means}
\begin{equation*}
    NL[I](p) = \frac{1}{W_p} \sum_{q\in S} G_{\sigma_r}(||\vec{V}_p-\vec{V}_q||^2)I_q
\end{equation*},
where $V_p$ is a vector of pixel values in an arbitrarily sized neighbourhood (controlled as a parameter) with centre in $p$.
\begin{equation*}
    W_p = \sum_{q\in S} G_{\sigma_r}(||\vec{V}_p-\vec{V}_q||^2)
\end{equation*}

\subsubsection{Cauchy distribution}
\begin{equation*}
    \frac{1}{\pi \gamma \left[1+\left(\frac{x-x_0}{\gamma}\right)^2\right]}
\end{equation*}