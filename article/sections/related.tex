\section{Related work}
This work aims to improve the status quo and provide a more comprehensive overview of iris obfuscation methods used in research so far. Related work therefore includes other obfuscation methods as well as methods in image and signal processing, information theory, and differential privacy.

As mentioned, obfuscation of iris patterns has previously been attempted (REFS). (REF) presented the initial claim that a low-pass filter would be able to selectively obfuscate the iris pattern while still being able to detect eye features (pupil and corneal reflection) in the image. This is based on the notion that an image $I$ can be decomposed into an iris component $I_R$ and a feature component $I_C$ such that $I=I_R+I_C$. This has inspired the communication model and use of mutual information in this paper although we make several extensions to allow its use in evaluation of the proposed iris obfuscation methods. Additionally, we challenge the study's claim that $I_C$ is dominated by low frequencies in the frequency domain by demonstrating how the bilateral and non-local-means filter which are selective low-pass filters, both outperform Gaussian blurring in our tests. A later study (REF) uses a salt-and-pepper like filter named \emph{snow} to perform iris obfuscation which is also included in the tests presented here. The study also presents a model for how the method could be implemented in hardware and thus shares our sentiment of the importance of isolation.

Image 

