\section{Introduction}
It is well known that eye tracking data, including images and gaze signals, can be used in identification of mental and physical traits including a persons identity. This is cause for concern due to the increasing prevalence of products using eye tracking technologies and availability of eye tracking datasets.

\emph{Iris recognition} in particular is of concern due to the maturity, availability, and high accuracy of existing methods. The human iris is ideal for personal identification because it exhibits three very important characteristics. (1) The iris pattern which is created by complex interactions between light and the fibrous stroma is highly variable across large populations. (2) Although genes do affect the development of certain iris pattern characteristics, they are generally considered to develop randomly (or as a response to the environment) during development - therefore identical twins and even individual's left and right eyes differ significantly. (3) Iris patterns are stable throughout life. Iris recognition is one of the strongest known biometrics (ref) with state of the art studies reporting \emph{false acceptance rates} (type II error) of at most $2\times 10^{-10}$\cite{DAUGMAN_NEW}. Because head-mounted eye trackers typically capture images where the iris has a high resolution (caused by the eyes proximity to the camera), the images can be reliably used for iris recognition \cite{BRENDAN_BLUR}. 

The possibility of personal identification from eye tracking products has both potential legal and ethical implications. For example, the GDPR (general data protection regulation) regulation defines personal data as including ``any information relating to an identified or identifiable natural person'' \cite{eu-gdpr}. Here, \emph{identifiable} person is defined as ``any person who can be distinguished from others'' \cite{eu-gdpr}. In other words, iris patterns are covered by the GDPR since they can be used to discover the identity of a person. This has the potential to increase the difficulty and feasibility of using eye tracking in both research and consumer products due to the added complexity of handling personal data. From an ethical standpoint this is also a problem of users or participants knowing what information they are actually sharing about themselves. In this age of large-scale data collection, being able to very accurately link other behaviours to identities has to be taken seriously.

The iris pattern data itself is not necessarily directly used by the eye tracking system itself. Therefore, a possible solution to this set of problems is to alter the eye images in a manner that decreases the accuracy of iris recognition while causing minimal impact to the estimation of gaze. We use the term \emph{iris obfuscation} for this task. For applications in consumer and commercial products, implementations of obfuscation should ensure that the original, unmodified image can never be accessed. This is only possible if the obfuscation method is built into the source camera. Due to both physical limitations and cost considerations, practical obfuscation methods should therefore be as simple as possible. 

Currently, no comprehensive overview and comparison of existing and new methods exist. A number of studies using optical defocus (and Gaussian filters) \cite{BRENDAN_BLUR, BRENDAN_ARTICLE} and salt-and-pepper noise \cite{BRENDAN_SNOW} have shown promising results but lack large-scale testing of iris recognition performance and method comparisons. A comparison study does exist \cite{IRIS_OBF} but it is focused on methods that require detection of the iris itself which, for the purposes of this paper, is considered to be outside the reasonable expectations for cheap embedded implementations in eye tracking systems. Additionally, the study does not consider gaze estimation accuracy. 

This paper presents a comprehensive study of a number of existing and new methods for iris obfuscation that are all based on image filters and noise functions. Of the proposed methods, the \emph{bilateral filter} achieves the highest rate of obfuscation with a measured \emph{false rejection rate} of $??\%$ at a mean relative gaze error of $1$ compared to the baseline. This is xx lower than xx and so on. Additionally, we propose the use of a communication model as the basis for understanding and evaluating the behaviour of iris obfuscation methods without relying on specific implementations for iris recognition and gaze estimation. Finally, we discuss the actual level of anonymity the shown methods provide depending on the use case. This also includes an examination of the differential privacy based approach in\cite{BRENDAN_SNOW}.

\subsubsection{Overview}
This work is comprised of two experiments. Both use individual datasets for gaze estimation, pupil detection, and iris recognition. The first experiment concerns discovery of optimal parameters for the presented obfuscation methods. and how small parameter changes impact gaze accuracy and iris recognition. It uses smaller samples from each dataset to estimate a number of metrics for each parameter configuration. The results provide insight into how the methods affect the images and how these changes are connected to each metric. Additionally, the results are used to select a small number of interesting configurations for further testing. The second experiment tests these configurations on the entire datasets. The results of this experiment includes a complete cross-comparison of $2,226$ samples from the CASIA IV (REF) dataset. 