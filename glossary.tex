\makenoidxglossaries

\newacronym{por}{PoR}{Point of Regard}
\newacronym{hci}{HCI}{human-computer-interaction}
\newacronym{vr}{VR}{virtual reality}

\newglossaryentry{eye-process}
{
    name={eye information process},
    description={Is any process that uses a human eye as its input and has the aim of extracting some property from its appearance or movements}
}

\newglossaryentry{eye-tracking}
{
    name=eye-tracking,
    description={Is the processes of detecting the location and movements of eyes and estimating \emph{gaze}}
}
\newglossaryentry{eye-tracker}
{
	name=eye-tracker,
	description={a hardware and software system that performs \gls{eye-tracking}}
}
\newglossaryentry{gaze}
{
    name=gaze,
    description={either as a specific \acrfull{por} or as a direction}
}

\newglossaryentry{scene-camera}
{
	name=scene-camera,
	description={camera that is fixed to the head of the subject and thus represents an approximation of their view at a given point in time}
}

\newglossaryentry{head-mounted}
{
	name=head-mounted,
	description={a type of \gls{eye-tracker} where the eye-facing camera is fixed relative to the head of the subject, hence translational movements of the eye are kept to a minimum}
}

\newglossaryentry{eye-system}
{
	name=eye information processing system,
	description={a system which attempts decoding of human properties through human eye images}
}




%% This code creates the groups
% -----------------------------------------
\usepackage{etoolbox}
\renewcommand\nomgroup[1]{%
  \item[\bfseries
  \ifstrequal{#1}{E}{Eye processing model sets}{%
  \ifstrequal{#1}{T}{Terminology}{%
  \ifstrequal{#1}{O}{Other Symbols}{}}}%
]}
% -----------------------------------------

\makenomenclature

% \nomenclature[E]{$\mathcal{Q}$}{Set of input signals in an eye processing model.}
% \nomenclature[E]{$\mathcal{R}$}{Set of decoded results.}
% \nomenclature[E]{$\mathcal{f_e}$}{Encoding function.}
% \nomenclature[E]{$\mathcal{P}$}{Set of processing functions.}
% \nomenclature[E]{$\mathcal{D}$}{Set of decoders.}
% \nomenclature[E]{$Q_i$}{Specific input signal.}
% \nomenclature[E]{$R_i$}{Specific output signal corresponding to the transmitted version of $Q_i$.}
% \nomenclature[E]{$f_e$}{Encoding function.}
% \nomenclature[E]{$f^p_i$}{Specific processing function.}
% \nomenclature[E]{$f^d_i$}{Specific input signal.}

\nomenclature[O, Z]{$P_{X}(x)$}{Probability density function of $X$.}
\nomenclature[O, G]{$\delta_{x,y}$}{Dirac delta function.}
\nomenclature[O, A]{$H(X)$}{Entropy of $X$.}
\nomenclature[O, A]{$\mathcal{I}(X;Y)$}{Mutual information of $X$ and $Y$.}
